\item {\bf Canine ECG Data} 

We have finally reached the point in the
course where we can start working with some real biosignals!  Working with real
signals means that we no longer have ``simple'' analytic expressions to
represent our input signals... so we are going to rely on numerical methods to
evaluate our signals.  Be sure to include all of your code for this problem.

The first signal that we will be working with is an ECG trace from a canine
cardiac study.  The dog had undergone several radiofrequency ablations before
this trace was acquired, and it experienced some degree of myocardial ischemia
(i.e., this is not a normal ECG trace).  The ECG data (\verb+ecg+) is in
millivolts, and the time variable (\verb+t+) is in seconds.

\begin{enumerate}
\item Download \verb+BME354L_CANINE_ECG.mat+ from Sakai.

\item Load \verb+BME354L_CANINE_ECG.mat+ into your Matlab workspace.

\item Plot the ECG signal as a function of time, with the axes properly labeled
with units.  Print this plot out to turn in with your problem set.

\item What is the range of voltages for the ECG signal?  What is the
approximate voltage range for the noise?\footnote{There are several different
ways that you can quantify these values that are equally valid but may provide
different values.  Clearly state/show the way that you estimated these values.}

\item What is the minimum number of bits that we need to accurately represent
the data?  What was the approximate SNR for this ECG trace?

\item What should the voltage of the least-significant-bit (LSB) be set to in
your ADC? 

\item What is the sampling frequency ($f_s$) for these data? 

%\item Implement a sliding window averager to improve the SNR of the ECG data.
%Plot the (reduction in) noise (just the noise, not the entire ECG signal) as a
%function of 5 different window sizes.  Again, include your code with your
%plots.

\item Generate a plot clearly showing the frequency content of this ECG data trace. There are
several ways to do this in Matlab; the most direct way involves using the
\verb+fft+ and \verb+fftshift+ commands.\footnote{If you need
to brush up on your FFT background from BME171, then there is a great online
reference: \url{http://www.dspguide.com/}; the chapters are freely available
online.} 

\item Next, we're going to manually downsample the data to qualitatively get a
feel for what minimum sampling frequency can be to still accurately represent
the morphology of the ECG signal.  Downsample the ECG data (and the time data)
by factors of 10, 50, 100, 200, 500, and 1000.  You can do this in Matlab by
skipping data points (e.g., you can create a vector of every tenth sample in
the ECG signal using the syntax \verb+ecg(1:10:end)+).

Generate plots comparing these waveforms (zoom in on just one of the wave
complexes) and notice the aliasing / signal corruption that occurs when the
waveform is undersampled.  Print these plots to hand in with your problem set
(you can put them all on the same plot as long as you use different line styles
to distinguish them; the \verb+subplot+ command can also be used to save
space/pages). 

\item Generate plots of the frequency content of these downsampled ECG data
traces and comment on what you expected these to look like relative to the
original signal and if your plots match your intuition. 

\item Using this manual technique to visually evaluate the effects of aliasing,
what is the minimum sampling frequency that could be used to adequately
represent these data digitally?  What does this imply about the maximum
frequency content of the ECG signal?  How does this compare with the Nyquist
limit that you would have chosen based on your power spectrum from Problem Set
\#6? 

\item How much greater than this minimum sampling frequency was the data
acquisition sampling frequency? Is this adequate? 

%    \item Using any of the techniques covered in class to reduce the noise in
%signals (e.g., filtering, averaging), evaluate the power spectrum of your
%``cleanest'' (i.e., highest SNR signal), which preserves all of the features of
%the ``true'' signal, with the power spectrum of the original signal.  Comment
%on the frequency content of these two signals relative to the sampling
%frequency and your conclusions above with respect to the Nyquist frequency. Be
%sure to include all of your code! [2 points]

\item Many datasets in research and industry are saved as ``raw'' binary files,
as opposed to ASCII files (too large) or proprietary formats (such as Matlab's
\verb+*.mat+).  Download \verb+BME354L_CANINE_ECG.dat+ from Sakai; this
data file contains a modified version of the ECG data that you had in the Matlab file.
The data are formatted as 32-bit floating point numbers and were saved as a
serial stream of time/voltage pairs (e.g., t1,v1,t2,v2,\ldots).  Read these data
in Matlab or your program of choice (something that will let you perform Fast
Fourier Transforms and cross correlations)\footnote{In addition to Matlab, you
can also use R, Excel, LabVIEW, Maple, Mathematica, or Octave\ldots it is your
choice.}.  In Matlab, commands that may make your life easier are \verb+fopen+,
\verb+fseek+, \verb+fread+, and \verb+fclose+ (and yes, this is more work than
just loading in the \verb+*.mat+ file, but much more generalized).  What are
the differences between this ECG signal and the original Matlab signal?
(\verb+BME354L_CANINE_ECG.mat+)?  Include your code and plots with your
problem set. 

\item Compute the power spectrum for this new ECG signal and compare/contrast
it with the power spectrum for the original ECG signal.  What would you need to
do to this new ECG signal to achieve the same power spectrum as the original
ECG signal?  

\end{enumerate}
