\documentclass[10pt]{article}
\usepackage{wrapfig}
%\usepackage{epsf}
\usepackage{fancyhdr,epsfig,graphics,tabularx,times}
\usepackage{amsmath}
\usepackage{amssymb}
\usepackage{palatino}
%\usepackage[dvips]{graphics}
\usepackage{fancyhdr}
\usepackage{epsfig}
\usepackage{multirow}
\usepackage{cancel}
\usepackage[bookmarks]{hyperref}
\usepackage{longtable}
\usepackage{soul}
\parindent 0in
\parskip 1ex
\oddsidemargin  0in
\evensidemargin 0in
\textheight 8.5in
\textwidth 6.5in
\topmargin -0.25in
\setcounter{section}{0}

\pagestyle{fancy}
\lhead{\bf BME354L: Introduction to Medical Instrumentation}
%\rhead{\bf Nightingale (Spring 2015)}
%\cfoot{\thepage}
\fancyfoot[C]{Document created by Dr. Mark Palmeri.  This work is licensed under a Creative Commons Attribution 3.0 Unported License.}




\begin{document}
\section*{Problem Set \#5: Quadrature \& Phase Demodulation}

\textbf{DUE:} Wednesday, 2014-03-19 at 5:00 PM \textbf{as a Sakai assignment
    attachment}.  Attach all code used to implement your algorithms and
    generate your plots!

\begin{enumerate}

\item \textbf{Quadrature Demodulation}

    We introduced the RLC divider in PS 3 as a circuit to detect changes in capacitance.  Referring to the RLC circuit in PS 3,

    \begin{enumerate}
        \item Plot several cycles of an input sinusoid at a frequency of 1000
            krad/s with a peak-to-peak amplitude of $\pm$ 2 V.  Plot the output
            signals from your RLC divider for capacitances of 0.95 nF and 1.05
            nF ($\pm$ 0.05 nF from the notch of the divider) on the same plot.

        \item Express each output signal in decibels relative to the input
            amplitude.

        \item Plot 10 cycles of an input sinusoid signal that linearly
            increases in amplitude from $\pm$ 2 V to $\pm$ 4 V over 10 cycles.  
            Plot the same output signals as in (a) for this input signal.

        \item Using the quadrature demodulation algorithm presented in lecture,
            plot the in-phase (I) and quadrature (Q) signals of each output
            signal from (c) on the same plot, with a separate plot for each
            output signal.  Add a third line to each plot that represents the
            envelope of the output signal, as computed from I and Q.

    \end{enumerate}

\item \textbf{Phase Demodulation}

    Problem 1 should convey the fact that the magnitude of the amplitude is
    related to the capacitance change, but this quantity does delineate if the
    capacitance has increased or decreased relative to the notch capacitance.
    The sign of the phase of the output signal relative to the input signal
    will indicate this increase / decrease.

    \begin{enumerate}
        \item Using the phase demodulation present in lecture, calculate the
            sign of the relative phase of the output compared to the input
            signal for each capacitance change in Problem 1(a).  
            
        \item Your algorithm in 2(a) should have included low pass filter,
            which you could have implemented numerically with an algorithm of
            your choice.  Plot the output of your phase demodulator before and
            after this low pass filtering operation and comment on the
            effectiveness of its performance.

        \item Plot the phase change for each output signal for the linear
            amplitude change in the input signal in Problem 1(c).   Comment on
            this plot.

    \end{enumerate}
\end{enumerate}

\end{document}
