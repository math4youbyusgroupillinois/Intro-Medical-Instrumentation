\documentclass[10pt]{article}
\usepackage{wrapfig}
%\usepackage{epsf}
\usepackage{fancyhdr,epsfig,graphics,tabularx,times}
\usepackage{amsmath}
\usepackage{amssymb}
\usepackage{palatino}
%\usepackage[dvips]{graphics}
\usepackage{fancyhdr}
\usepackage{epsfig}
\usepackage{multirow}
\usepackage{cancel}
\usepackage[bookmarks]{hyperref}
\usepackage{longtable}
\usepackage{soul}
\parindent 0in
\parskip 1ex
\oddsidemargin  0in
\evensidemargin 0in
\textheight 8.5in
\textwidth 6.5in
\topmargin -0.25in
\setcounter{section}{0}

\pagestyle{fancy}
\lhead{\bf BME354L: Introduction to Medical Instrumentation}
\rhead{\bf Nightingale (Spring 2015)}
\cfoot{\thepage}



\begin{document}
\section*{Problem Set \#2: Amplifiers and Comparators}

\textbf{DUE:} Friday, 2013-01-31 at 5:00 PM in the grader box

\begin{enumerate}

\item {\bf Instrumentation Amplifier} A graduate student at Duke is working on
    measuring axon action potentials that range in voltage from -100 mV to +50
    mV.  This student needs to amplify these signals before feeding them into
    additional signal processing equipment that will quantify the time
    constants associated with the axon firing.  Design a three-op-amp
    differential amplifier (i.e., an instrumentation amplifier) that has a
    differential gain of 5 in the first (buffering) stage and a differential
    gain of 6 in the second (diff amp) stage.  Clearly draw your circuit with
    all of the circuit elements and their values clearly labeled. \\

    Based on the specified range of the input signal, what are the minimum
    limits of the rail voltages of each op-amp to insure that the cumulative
    output of the amplifier remains in the linear range?\\

\item {\bf Summing Amplifier} Ultrasound transducers are composed of many
    different piezoelectric elements, each of which is independently controlled
    electronically.  In addition to using that electrical independence to focus
    the sound waves, it can also be used to control the amplitude of the sound
    wave sent and received from each element.  Controlling these element
    amplitudes can be used to apodize the acoustic aperture to reduce side
    lobes of the focused sound wave, increasing the system's spatial
    resolution.  This side lobe reduction is achieved by reducing the received
    voltage amplitudes of the elements on the ends of the aperture relative to
    the center elements before they are summed together.  For a 5 element
    ultrasound transducer ($E_{1-5}$), design a summing amplifier so that two
    elements ($E_{2,4}$) adjacent to the center element ($E_3$) are 85\% the
    center element's amplitude, and the outer elements ($E_{1,5}$) are 25\% the
    center elements excitation voltage (i.e., $v_o = v_{E_3} + 0.85 v_{E_2} +
    0.85 v_{E_4} + 0.25 v_{E_1} + 0.25 v_{E_5}$).\\

\item {\bf Comparator} Design a comparator with a hysteresis range from 0 to +2
    V.  Make sure that you indicate the value of $V_{ref}$.  Assume that the
    rail voltages of the op-amp are $\pm$ 13 V.\\

    Sketch the output of your comparator over three wavelengths of an input
    signal $v_{in}$ = 4 $\sin(\omega t + \pi/4)$, for a frequency of 125.66
    rad/s (20 Hz), starting at $t$ = 0.  (Sketch the output and input on the
    same set of axes.) \\

\item {\bf Design: Pulmonary Hypertension Alarm} An anesthesia monitoring
    system is measuring the pressure output from a Swan-Ganz pulmonary artery
    catheter to evaluate pulmonary hypertension intraoperatively.  Since the
    anesthesiologist is also monitoring several other physiologic systems
    concurrently (e.g., ECG, EEG, respiratory rate, blood pressure, etc.), the
    pressure monitor sounds an alarm when the pressure goes too high or too
    low.  The pressure output of the catheter, ranging from 0 - 100 mm Hg, is
    linearly mapped to voltage through its transduction mechanism to be 0 - 10
    V.  Design a circuit using a comparator that will sound the alarm when the
    pressure rises above 25 mmHg.  Assume that the pressure catheter signal has
    a noise component of $\pm$ 0.1 mmHg; your comparator should have a
    hysteresis range that is twice this noise limit.  Let your input signal to
    the circuit from the pressure catheter be represented as $v_{in}$.  Use DC
    power sources as needed in your circuit design, and specify the rail
    voltage(s) for the op-amp(s).  The alarm can be considered a ``black box''
    load that can be activated with a voltage greater than 5 V ($v_o >$ 5 V).\\

\item {\bf More Design: Low \& High} Using a second comparator, modify your
    circuit in the previous problem so that the alarm will sound when the
    pressure rises about 25 mmHg or falls below 10 mmHg.  You want to use the
    same alarm for both conditions, so you will need to use another type of
    circuit component we discussed in class to combine the outputs of the two
    comparators to drive the alarm. \\

\item {\bf Rechargeable, battery-powered devices} are becoming more ubiquitous
    in the medical setting.  Compare/contrast these four types of rechargeable
    batteries in terms of energy density, output voltage, battery life,
    recharge time, and weight: (1) lead acid, (2) NiCad, (3) NIMH, and (4)
    Li-ion.

    Li-ion batteries have become very common in cell phones, digital cameras,
    laptop computers, and many other portable electronic devices.
    Unfortunately, there is also a non-negligible fire risk associated with
    these devices.  What is responsible for this fire risk, and what safeguards
    do modern versions of these batteries utilize?

\end{enumerate}

\end{document}
