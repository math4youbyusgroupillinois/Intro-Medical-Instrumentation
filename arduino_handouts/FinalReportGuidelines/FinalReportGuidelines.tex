\documentclass[10pt]{report}
\usepackage{epsf}
\usepackage{amsmath}
\usepackage{amssymb}
\usepackage{palatino}
\usepackage[dvips]{graphics}
\usepackage{fancyhdr}
\usepackage{epsfig}
\usepackage{multirow}
\usepackage{cancel}
\parindent 0in
\parskip 1ex
\oddsidemargin  0in
\evensidemargin 0in
\textheight 8.5in
\textwidth 6.5in
\topmargin -0.25in

\pagestyle{fancy}
\fancyhead[LO]{\bf BME354L - Palmeri - Spring 2013}
\fancyhead[RO]{{\bf Arduino Project: Final Report Guidelines}}

\begin{document}

After completing the basic Arduino tutorials, the LCD \& Control System lab
exercises, and reviewing the functional specifications of your final Arduino
surface mount reflow oven specification in the Reflow Tutorial, you should have
demonstrated the function of your Arduino-based reflow oven controller using
the thermocouple and heater setup provided by Dr. Palmeri\footnote{Designed and
constructed by Will Scheideler, Oliver Fang, and Matt Brown.}.  After
successfully demonstrating the function and features of your device, you are
expected to submit a single, formal written report for your group that builds
upon all of the lab report experience you have gained throughout the semester.
Your final report will also function as a user manual and troubleshooting
guide.  Please be sure to include the items below in your final report; these
are the minimum expectations, and additional items can be included as you feel
appropriate.

\begin{itemize}

  \item Introduction: provide an overview of the function of a reflow oven, and
    the functional need for your Arduino microcontroller in this project

  \item Background: provide factual information about any of the pieces of this
    project that would be useful for the reader to know to interpret items
    later in your report

  \item Materials \& Methods: provide an overview of the materials and methods
    used in this project.  Please include information about the thermocouple
    stage (including linearizing circuit) and the SSR / heater stage that is
    appropriate from the context of interfacing with your Arduino
    microcontroller.  You can reference all of your Arduino code in an
    appendix.  It may be useful to include flow charts of your code structure,
    your overall device function, and your user interface.

    \item Results: provide an overview of how well your device performed in the
      context of the project demonstration

    \item Discussion: provide a discussion of your overall approach, and in
      hindsight, critical analysis of anything that you would change to better
      meet the functional specifications of your device.  Also highlight any
      significant learning points for your group during the development of the
      functional microcontroller.  If you used GitHub for your project code
      management, then please comment on your collective group experience using
      this version control software tool.\footnote{Negative feedback will not
      be held against you!}

    \item Conclusions

    \item Appendices
      \begin{itemize}
            \item Arduino microcontroller code, {\bf commented}.  Please include a URL to your GitHub repository if one was utilized throughout the project.
            \item Simple user manual: if someone were to turn on the reflow
              oven with your code already uploaded to the microcontroller, then
              how would they use the device
            \item Troubleshooting guide: if you know that your microcontroller code has quirks, then please provide a Troubleshooting table of common problems and how to remedy them (if they can be remedied by the end user)
      \end{itemize}

\end{itemize}

As a general tip when preparing this report, \emph{well-done} figures, flow
charts and tables can be used to effectively replace pages of text.  Please be
sure to include all references to sources utilized throughout your project
development and analysis.

\end{document}
