\section{Lab Policies \& Reports}
\subsection{Policies}
\begin{itemize}
    \item Laboratory work is worth 25\% of your final grade for this course.
        This includes your pre-lab assignments, active participation during
        lab, appropriate data recording / documentation in your Sakai group lab
        folder, and your final written report for each lab.

    \item Labs are due at the beginning of the next scheduled lab, as specified
        by the Sakai assignment. Reports will be \textbf{electronically
            submitted via Sakai}.  Several lab reports may be combined with
        similar experiments on subsequent weeks; these specific reports will be
        clearly identified in class and lab, and are indicated on the class
        schedule.  When in doubt, please check Sakai to determine when a lab
        report is due.

    \item Late lab reports will be penalized 50\% of otherwise earned credit
        for being up to one day late.  Lab reports greater that one day late
        will receive no credit, but must be completed and turned in to pass the
        class.  {\bf Late lab reports must be submitted through Sakai, which
            Dr. Palmeri will have to give you access to do after the original
            due date.}

    \item You must inform Dr. Palmeri if you have a known conflict with
        attending a lab \emph{at least one week before that lab}.  \textbf{Two
            unexcused absences from lab will result in an automatic failure of
            the course.}

    \item \textbf{You cannot attend the lab section that you are not enrolled in!}

    \item All laboratory exercises and reports must be completed to pass this
        course.
\end{itemize}

\subsection{Pre-lab Assignments, Sakai Data Storage / Documentation, Lab Procedures}
\begin{itemize}
    \item All pre-laboratory questions and calculations must also be completed
        \underline{before} lab and electronically submitted before the start of
        lab (one per group). Additionally, for some labs, you will be asked to
        complete Arduino code before lab.  Lab-specific details will be
        provided when appropriate.  These pre-laboratory assignments are with
        10\% of each lab's grade, and late pre-laboratory submissions will not
        receive any credit.
    \item Electronically record notes of your procedures and results throughout
        the lab using the PCs in the laboratory, and upload all electronic
        documents (text files, spreadsheets and screen captures) to your Sakai
        account so your TAs can review this materials if necessary.  \emph{Make
            sure that your notes have adequate details to complete your lab
            report, including all necessary data and screen captures.}
    \item You and your lab partner must clear your lab bench space and
        return all parts after you have completed your experiment.  
\end{itemize}

\subsection{Lab Reports}
\begin{itemize}
    \item Each lab \underline{group} (i.e., you and your lab partner) must
        complete a lab report for each lab.  Like performing the lab exercise,
        the writing of the report and the data analysis should be even between
        the lab partners.
    \item The lab report is not required to be a certain length, but should be
        long enough to adequately present results, discuss those results
        (including things that do not make sense, and why that may be the
        case), and appropriately answer any questions raised during the
        procedure and in the post-lab questions.  Distilling your information
        down to the pertinent points is more important than including the
        ``kitchen sink'' in your lab reports.
    \item The following is a general outline for your lab reports (your lab TAs
        will discuss their expectations in more detail during lab):
    \begin{itemize}
        \item \underline{Purpose}: State the purpose of the experiment in one
            or two sentences
        \item \underline{Materials \& Methods}: List the material and equipment
            used during the lab (e.g., Oscilloscope Tektronix TDS-1012).  You
            do not need to repeat the procedure already mentioned in the lab
            handout; instead, please include any deviations from the protocol,
            reasons for those deviations, etc.  After reading this section of
            the lab report, someone should be able to replicate what you did
            along with lab handout.
        \item \underline{Results \& Discussion}: Provide all of the data
            acquired in the lab in a meaningful and efficient format (e.g.,
            tables, plots, etc.).  All figures, tables, plots \emph{must} be
            numbered and labeled, with captions, and properly referenced in the
            text.  All figure / plot axes must be labeled and include units!!
            All results must include units!! Included all analyses discussed in
            the lab handout, and provide equations that were used in these
            analyses (and any intermediate steps if they are not obvious and
            are significant).  Answer all questions posed in the lab handout
            and post-lab questions (be sure to indicate the question number you
            are answering).  
        \item \underline{Conclusions}: A few sentences providing an overview of
            your findings, interesting observations, and overall success of
            your experiment.
    \end{itemize}
    \item All reports must be generated using a word processor (e.g., Word) or
        typesetting program (e.g., \LaTeX) using a reasonable font size and
        single-spaced with clear section headings.
    \item Please include a title page with:
        \begin{itemize}
            \item Lab title
            \item Date
            \item Section number
            \item TA name
            \item Your name and the names of your lab partners
            \item Declaration of adhering to the principles of the Duke Community Standard while preparing your lab report
        \end{itemize}
    \item Please number each page as `Page \# of \#' so your TA knows how many
        pages should be in your lab report.
    \item {\bf PDF electronic versions of your lab reports will be submitted to
            Sakai via the specific Lab Assignment that you and your lab partner
            will have access to.  Only one lab report needs to be submitted for
            each group.}
\end{itemize}
