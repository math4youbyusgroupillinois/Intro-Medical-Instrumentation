\documentclass[10pt]{report}
\usepackage{epsf}
\usepackage{amsmath}
\usepackage{amssymb}
\usepackage{palatino}
\usepackage[dvips]{graphics}
\usepackage{fancyhdr}
\usepackage{graphicx}
\parindent 0in
\parskip 1ex
\oddsidemargin  0in
\evensidemargin 0in
\textheight 8.5in
\textwidth 6.5in
\topmargin -0.25in

\pagestyle{fancy}
\fancyhead[LO]{\bf BME154L - Palmeri - Spring 2011}
\fancyhead[RO]{\bf Exam \#2 Topics}
\fancyfoot[C]{\thepage}

\begin{document}
{\bf Topics to Review:} The following topics are fair game on the second exam.
Remember, you are allowed to bring a review sheet to the exam that fills one
side of an 8.5x11'' piece of paper.  Also remember to bring a calculator.

\begin{itemize}
\item First exam topics (not to the same level of detail, but you should still
remember how to analyze and design circuits).
\item Input/output impedance (too many people missed this on the first exam)
\item Signal Processing
\begin{itemize}
    \item SNR
    \item Noise (types, frequency characteristics)
    \item Noise reduction approaches
    \begin{itemize}
        \item Filtering
        \item Coherent temporal averaging
        \item Non-running, running and exponential averagers
        \item Theoretical SNR improvements
        \item Block diagrams describing averaging algorithms
        \item Correlation
    \end{itemize}
    \item Frequency-domain Analysis
    \begin{itemize}
        \item Fourier transform pairs for ``common'' functions, including delta functions, rects, sinusoids, combs, Gaussians, etc.
        \item General properties of the Fourier Transform, including those outlined in the lecture handout
    \end{itemize}
    \item Convolution
    \item Auto- and Cross-Correlation (properties of, how to perform the operation, why it is useful, etc.)
\end{itemize}
\item Digital Electronics
\begin{itemize}
    \item Digital logic gates
    \item Combinatorial and sequential logic
    \item SR, D and JK flip flops
    \item Registers, latches, counters and timing diagrams
    \item Binary numbers
    \item Analog $\rightarrow$ Digital
    \begin{itemize}
        \item Bit resolution
        \item Sampling rates and aliasing
        \item Flash ADC
        \item Successive Approximation ADC
        \item Single-Slope Integration ADC 
    \end{itemize}
    \item Digital $\rightarrow$ Analog
    \begin{itemize}
        \item Resolution
        \item Scaled-resistors into summing amplifier
        \item R-2R Ladder
    \end{itemize}
\end{itemize}
\item Cardiovascular System
\begin{itemize}
    \item Heart and vascular anatomy; conduction anatomy
    \item ECG signal
    \begin{itemize}
        \item What does it represent electrically?
        \item How is it measured?
        \begin{itemize}
            \item Eindhoven's triangle
            \item Wilson's Central Terminal
            \item $>$ 3-lead configurations
        \end{itemize}
        \item How does it related to other physiologic processes (e.g., contraction, blood pressure, etc.)
        \item Sources of noise and methods of minimization / compensation
    \end{itemize}
    \item Heart conduction abnormalities (what they are and how they manifest themselves in measurement systems)
    \item Arrhythmias (what they are, how to diagnose, how to treat)
    \item Pacemakers
    \end{itemize}
\end{itemize}

{\bf Types of questions to expect:}  This exam will be different from the
first exam.  Expect questions that focus on block diagram design, understanding
the physiologic systems mentioned above in the context of characterizing them
(which includes knowing some factual information about them), and biosignal
processing.  You will be asked to evaluate a circuit in terms of what it does in
the context of making a measurement.

\end{document}
