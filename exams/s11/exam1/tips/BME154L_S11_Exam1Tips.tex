\documentclass[10pt]{report}
\usepackage{epsf}
\usepackage{amsmath}
\usepackage{amssymb}
\usepackage{palatino}
\usepackage[dvips]{graphics}
\usepackage{fancyhdr}
\usepackage{graphicx}
\parindent 0in
\parskip 1ex
\oddsidemargin  0in
\evensidemargin 0in
\textheight 8.5in
\textwidth 6.5in
\topmargin -0.25in

\pagestyle{fancy}
\fancyhead[LO]{\bf BME154L - Palmeri - Spring 2011}
\fancyhead[RO]{\bf Exam \#1 Tips}
\fancyfoot[C]{\thepage}

\begin{document}
{\bf Topics to Review:} The following topics are fair game on the first exam.
Remember, you are allowed to bring a review sheet to the exam that fills one
side of an 8.5x11'' piece of paper.  Also remember to bring a calculator.

\begin{itemize}

\item Transducers
\begin{itemize}
    \item Resistive Transducers (e.g., strain gages, thermistors)
    \item Differential Capacitors
    \item LVDTs
    \item Thermocouples
\end{itemize}

\item Detection Circuits
\begin{itemize}
    \item Wheatstone Bridge
    \item Reactance Bridge
    \item RLC Divider
    \item What affects sensitivity, error reduction, etc.
    \item Phase Demodulators
    \item Full-wave Rectifiers
\end{itemize}

\item Amplifiers
\begin{itemize}
    \item Ideal Op Amp Equiv. Model
    \item Buffers
    \item Inverting, Non-inverting
    \item Summing
    \item Differential, Instrumentation
\end{itemize}

\item Passive and Active Filters
\begin{itemize}
    \item LPF, HPF, Bandpass, Bandstop
    \item Identify and derive transfer functions and cutoff/resonant frequencies for first- and second-order filters
    \item Bode plots (sketch and interpret)
    \item Design ``simple'' filters
    \item Integrators and differentiators
\end{itemize}

\item Comparators (with hysteresis)

\item Diodes, Relays and Transistors (to the level covered in lab and problem sets; used for isolation, switching) 

\item Miscellaneous
\begin{itemize}
    \item Know the difference between zero-order, first-order and second-order systems
    \item Input/output impedance
    \item Noise (different types, ways to actively and passively reduce noise at different stages in circuits)
    \item Resolution, sensitivity, accuracy, precision, SNR
\end{itemize}

\item Averagers, Convolution

\end{itemize}

{\bf Types of Questions to Expect}
\begin{itemize}
    \item Evaluate a circuit
    \begin{itemize}
        \item Solve for voltages and currents
        \item Sketch waveforms of outputs or mid-circuit nodes
        \item Modify a circuit to change its behavior
        \item Early errors will not hose you, but you must show all of your work!
    \end{itemize}
    \item Design a circuit
    \begin{itemize}
        \item Block diagrams are your best friend!
        \item Always state your assumptions
        \item More than one way to do it
    \end{itemize}
    \item Explain / define something
    \item Lab-based questions (not minutiae)
    \item Things {\bf NOT} on the exam:
    \begin{itemize}
        \item There will be no ``plug-and-chug'' problems; think homework problems with the ante upped a bit.
        \item There will be no LabVIEW questions.
        \item Correlation
        \item Digital
    \end{itemize}
\end{itemize}
\end{document}
