\documentclass[11pt]{article}
\usepackage{fancyhdr,epsfig,graphics,tabularx,times}
\pagestyle{empty}
\topmargin=-0.75in
\oddsidemargin=-0.375in
\textwidth=7.0in
\textheight=9.5in
\parindent=0.0in
\parskip=10pt
\linespread{0.9}

\begin{document}
{\bf BME154L (Palmeri, Spring 2010)} \hfill
{\bf Exam \#1 Tips}

{\bf Topics to Review:} The following topics are fair game on the first exam.
Remember, you are allowed to bring a review sheet to the exam that fills one
side of an 8.5x11'' piece of paper.  Also remember to bring a calculator.

\begin{itemize}

\item Transducers
\begin{itemize}
    \item Resistive Transducers (e.g., strain gages, thermistors)
    \item Differential Capacitors
    \item LVDTs
    \item Thermocouples
    \item Piezoelectrics
\end{itemize}

\item Detection Circuits
\begin{itemize}
    \item Wheatstone Bridge
    \item Reactance Bridge
    \item RLC Divider
\end{itemize}

\item Amplifiers
\begin{itemize}
    \item Buffers
    \item Inverting, Non-inverting
    \item Summing
    \item Differential, Instrumentation
    \item CMRR
    \item Non-ideal op amp characteristics (e.g., bias current, slew rate, resistor selection)
\end{itemize}

\item Passive and Active Filters
\begin{itemize}
    \item LPF, HPF, Bandpass, Bandstop
    \item Identify and derive transfer functions and cutoff/resonant frequencies for first- and second-order filters
    \item Bode plots (sketch and interpret)
    \item Design ``simple'' filters
    \item Integrators and differentiators
\end{itemize}

\item Comparators (with hysteresis)

\item Relays and Transistors (to the level covered in lab and problem sets; used for isolation, switching and one-bit ADC)

\item Binary Numbers
\begin{itemize}
    \item You do \underline{not} need to know offset binary, 2's complement, octal, hexadecimal, BCD.
    \item You do need to know how to convert between base10 $\leftrightarrow$ base2.
\end{itemize}

\item Logic Gates \& Truth Tables

\item D \& JK Flip Flops

\item Shift Registers \& Counters
\end{itemize}

\hfill There is more on the next page...
\clearpage

{\bf BME154L (Palmeri, Spring 2010)} \hfill
{\bf Exam \#1 Tips}

\begin{itemize}
\item Analog-to-Digital Conversion
\begin{itemize}
    \item Saturation and quantization 
    \item Sampling frequency and aliasing
    \item One-bit ADC
    \item Flash ADC
    \item Successive approximation (general approach, not details of schematic)
\end{itemize}

\item Digital-to-Analog Conversion
\begin{itemize}
    \item Summer with scaled resistors
    \item R-2R Ladder
    \item Need for LPF
\end{itemize}

\item Miscellaneous
\begin{itemize}
    \item Know the difference between zero-order, first-order and second-order systems
    \item Input/output impedance
    \item Noise (different types, ways to actively and passively reduce noise at different stages in circuits)
    \item Resolution, sensitivity, accuracy, precision
\end{itemize}

\end{itemize}

{\bf Types of Questions to Expect}
\begin{itemize}
    \item Evaluate a circuit
    \begin{itemize}
        \item Solve for voltages and currents
        \item Sketch waveforms of outputs or mid-circuit nodes
        \item Modify a circuit to change its behavior
        \item Early errors will not hose you, but you must show all of your work!
    \end{itemize}
    \item Design a circuit
    \begin{itemize}
        \item Block diagrams are your best friend!
        \item Always state your assumptions
        \item More than one way to do it
    \end{itemize}
    \item Explain / define something
    \item There will be no ``plug-and-chug'' problems; think homework problems with the ante upped a bit.
    \item There will be no LabVIEW questions.
\end{itemize}
\end{document}
