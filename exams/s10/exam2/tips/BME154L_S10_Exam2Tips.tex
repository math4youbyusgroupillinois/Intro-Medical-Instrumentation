\documentclass[11pt]{article}
\usepackage{fancyhdr,epsfig,graphics,tabularx,times}
\pagestyle{empty}
\topmargin=-0.75in
\oddsidemargin=-0.375in
\textwidth=7.0in
\textheight=9.5in
\parindent=0.0in
\parskip=10pt
\linespread{0.9}

\begin{document}
{\bf BME154L (Palmeri, Spring 2010)} \hfill
{\bf Exam \#2 Tips}\\


{\bf Topics to Review:} The following topics are fair game on the second exam.
You are {\bf not} allowed to bring a review sheet to the exam.  Also remember
to bring a calculator.

\begin{itemize}
\item Signal Processing
\begin{itemize}
    \item SNR
    \item Noise (types, frequency characteristics)
    \item Noise reduction approaches
    \begin{itemize}
        \item Filtering
        \item Coherent temporal averaging
        \item Non-running, running and exponential averagers
        \item Theoretical SNR improvements
        \item Block diagrams describing averaging algorithms
        \item Correlation
    \end{itemize}
    \item Frequency-domain Analysis
    \begin{itemize}
        \item Fourier transform pairs for ``common'' functions, including delta functions, rects, sinusoids, combs, Gaussians, etc.
        \item General properties of the Fourier Transform, including those outlined in the lecture handout
    \end{itemize}
    \item Convolution
    \item Auto- and Cross-Correlation (properties of, how to perform the operation, why it is useful, etc.)
\end{itemize}
\item Cardiovascular System
\begin{itemize}
    \item Heart and vascular anatomy; conduction anatomy
    \item ECG signal
    \begin{itemize}
        \item What does it represent electrically?
        \item How is it measured?
        \begin{itemize}
            \item Eindhoven's triangle
            \item Wilson's Central Terminal
            \item $>$ 3-lead configurations
        \end{itemize}
        \item How does it related to other physiologic processes (e.g., contraction, blood pressure, etc.)
        \item Sources of noise and methods of minimization / compensation
    \end{itemize}
    \item Heart conduction abnormalities (what they are and how they manifest themselves in measurement systems)
    \item Arrhythmias (what they are, how to diagnose, how to treat)
    \item Devices to characterize the cardiovascular system
    \begin{itemize}
        \item Stethoscope
        \item ECG Monitor
        \item Blood pressure meters
        \item Flow meters
    \end{itemize}
    \item Devices to intervene
    \begin{itemize}
        \item Pacemakers
        \item Defibrillators
        \item Cardioverters
    \end{itemize}
    \item Blood pressure 
    \begin{itemize}
        \item Pressures in the body at different anatomic locations
        \item Methods to measure pressure
    \end{itemize}
    \item Blood flow
    \begin{itemize}
        \item Methods to measure flow
        \item Design considerations
        \item Ultrasonic flow meters (as covered in lecture)
    \end{itemize}
\end{itemize}
\item Respiratory System
    \begin{itemize}
        \item Respiratory anatomy
        \item Measuring gas exchange
        \item Spirometry
        \item Characterizing mechanical systems with circuit equivalents
    \end{itemize}
\item Biotelemetry
    \begin{itemize}
        \item General functional blocks to achieve
        \item Design considerations
        \item Solutions to inherent bottlenecks/problems/concerns
    \end{itemize}
\item Electrical Safety
\item Lab Topics
\begin{itemize}
    \item Second-order systems
    \item Pneumotach
    \item The functional components of the circuits used to make the measurements in lab
    \item ECG signal analysis (overlap with problem sets)
\end{itemize}
\end{itemize}

{\bf Types of questions to expect:}  This exam will be very different from the
first exam.  Expect questions that focus on block diagram design, understanding
the physiologic systems mentioned above in the context of characterizing them
(which includes knowing some factual information about them), and biosignal
processing.  You may be asked to evaluate a circuit in terms of what it does
(not solving for detailed voltages and currents like the first exam, but knowing how the signal is processed by different stages of a circuit) in the context of making a measurement.  

Solutions for all of the problem sets have been posted on Blackboard.

\end{document}
