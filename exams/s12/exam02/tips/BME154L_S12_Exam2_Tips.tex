\documentclass[10pt]{report}
\usepackage{epsf}
\usepackage{amsmath}
\usepackage{amssymb}
\usepackage{palatino}
\usepackage[dvips]{graphics}
\usepackage{fancyhdr}
\usepackage{graphicx}
\parindent 0in
\parskip 1ex
\oddsidemargin  0in
\evensidemargin 0in
\textheight 8.5in
\textwidth 6.5in
\topmargin -0.25in

\pagestyle{fancy}
\fancyhead[LO]{\bf BME154L - Palmeri - Spring 2012}
\fancyhead[RO]{\bf Exam \#2 Tips}
\fancyfoot[C]{\thepage}

\begin{document} {\bf Topics to Review:} The following topics are fair game on
the second exam, in addition to all of the material that was covered on the
first exam.  Remember, you are allowed to bring a review sheet to the exam that
fills both sides of an 8.5x11'' piece of paper \underline{and} you can bring
your review sheet from the first exam (or make up a new review sheet for that
material).  Also remember to bring a calculator.

\begin{itemize}
\item First exam topics 
\item Lab topics
\item Signal Processing
\begin{itemize}
    \item SNR
    \item Noise (types, frequency characteristics)
    \item Noise reduction approaches
    \begin{itemize}
        \item Filtering
        \item Coherent temporal averaging
        \item Non-running, running and exponential averagers
        \item Theoretical SNR improvements
        \item Block diagrams describing averaging algorithms
        \item Correlation
    \end{itemize}
    \item Frequency-domain Analysis
    \begin{itemize}
        \item Fourier transform pairs for ``common'' functions, including delta functions, rects, sinusoids, combs, Gaussians, etc.
        \item General properties of the Fourier Transform, including those outlined in the lecture handout
    \end{itemize}
    \item Convolution
    \item Auto- and cross-correlation (properties of, how to perform the operation, why it is useful, etc.)
\end{itemize}
\item Digital Electronics
\begin{itemize}
    \item Digital logic gates
    \item Combinatorial and sequential logic
    \item SR, D and JK flip flops
    \item Registers, latches, counters and timing diagrams
    \item Binary numbers
    \item Analog $\rightarrow$ Digital
    \begin{itemize}
        \item Bit resolution
        \item Sampling rates and aliasing
        \item Flash ADC
        \item Successive Approximation ADC
        \item Single-Slope Integration ADC 
    \end{itemize}
    \item Digital $\rightarrow$ Analog
    \begin{itemize}
        \item Resolution
        \item Scaled-resistors into summing amplifier
        \item R-2R Ladder
    \end{itemize}
    \item Know how to design these ADCs and DACs, not just analyze circuits given to you!
\end{itemize}
\end{itemize}

\emph{Note: I have decided to move the lab practical to the final lab of the
semester (Lab 11), not the week during the second exam.  This lab practical
will be part of your lab average, not your second exam grade.}
\end{document}
