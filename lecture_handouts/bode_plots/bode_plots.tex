%\documentclass[10pt]{report}
%\usepackage{epsf}
%\usepackage{amsmath}
%\usepackage{amssymb}
%\usepackage{palatino}
%\usepackage[dvips]{graphics}
%\usepackage{fancyhdr}
%\parindent 0in
%\parskip 1ex
%\oddsidemargin  0in
%\evensidemargin 0in
%\textheight 8.5in
%\textwidth 6.5in
%\topmargin -0.25in

%\pagestyle{fancy}
\fancyhead[LO]{\bf BME354L - Palmeri - Spring 2013}
\fancyhead[RO]{{\bf Second-Order Systems}}
\fancyfoot[C]{This work is licensed under a Creative Commons Attribution 3.0 Unported License.}

\documentclass[10pt]{article}
\usepackage{wrapfig}
%\usepackage{epsf}
\usepackage{fancyhdr,epsfig,graphics,tabularx,times}
\usepackage{amsmath}
\usepackage{amssymb}
\usepackage{palatino}
%\usepackage[dvips]{graphics}
\usepackage{fancyhdr}
\usepackage{epsfig}
\usepackage{multirow}
\usepackage{cancel}
\usepackage[bookmarks]{hyperref}
\usepackage{longtable}
\usepackage{soul}
\parindent 0in
\parskip 1ex
\oddsidemargin  0in
\evensidemargin 0in
\textheight 8.5in
\textwidth 6.5in
\topmargin -0.25in
\setcounter{section}{0}

\pagestyle{fancy}
\lhead{\bf BME354L: Introduction to Medical Instrumentation}
%\rhead{\bf Nightingale (Spring 2015)}
%\cfoot{\thepage}
\fancyfoot[C]{Document created by Dr. Mark Palmeri.  This work is licensed under a Creative Commons Attribution 3.0 Unported License.}




\begin{document}

% The next 7 lines were used when printing this section as a handout:
\setcounter{page}{1}
\setcounter{equation}{0}
\setcounter{figure}{0}
\begin{center}
\Large{\textbf{Bode Plots}}
\normalsize
\end{center}


Bode plots, named after Hendrick Bode, are log-log plots of magnitude vs.\ frequency and phase
vs.\ frequency. 

Specifically, a \textbf{magnitude} Bode plots presents
decibels magnitude vs.\ log frequency:

$20 \log_{10} |\frac{v_{out}}{v_{in}}(\omega)|$ vs.\
$\log_{10}(\omega / 2 \pi)$

A \textbf{phase} Bode plot plots the phase of $H(\omega)$ vs.\ log frequency.

The \textbf{logarithmic scale} allows presentation of a wide range of
frequency, as well as transforming the curves into a more readable
linear format. (Note that the frequency scale never reaches zero.)

\textbf{Principles:}

For a given transfer function $H(\omega)$:
\begin{itemize}
\item When $|H(\omega)|~\alpha~\omega$, the slope of the Bode plot $= +20$
dB/decade.
\item When $|H(\omega)|~\alpha~1/\omega$, the slope of the Bode plot $= -20$
dB/decade.

\item Each $\omega$ or $j\omega$ in the numerator contributes $+20$
dB/decade.
\item Each $\omega$ or $j\omega$ in the denominator contributes $-20$
dB/decade.

\item Each term of form $1 + j \omega / \omega_c$ in the numerator is called
a \textit{zero}, and contributes $= +20$
dB/decade magnitude and a $+\pi/2$ phase shift above $\omega_c$.

\item Each term of form $1 + j \omega / \omega_c$ in the denominator is called
a \textit{pole}, and contributes $= -20$
dB/decade magnitude and a $-\pi/2$ phase shift above $\omega_c$.
\end{itemize}

\textbf{To sketch the Bode plots,} evaluate the transfer function at
the extremes of $\omega=0,~\omega=\infty $ and at each cut-off
frequency(s) $\omega_{c}$.

Connect these points with line segments. 

At $\omega_{c}$, $|H(\omega)|$ will be down 3 dB relative to the
pass-band, while the phase plot passes through $\pm 45$ degrees
relative to the pass-band.

\textbf{For example}, a low-pass filter has the transfer function:
\begin{equation}
\begin{split}
\frac{v_{out}}{v_{in}}(\omega)&= -\frac{Z_{f}}{Z_{i}}=
-\frac{R_{f}||C}{R_{i}}\\ &= -\frac{\frac{(R_{f}/j\omega
C)}{(1/j\omega C)+R_{f}}}{R_{i}}\\ &= -\frac{R_{f}}{(1+j\omega
R_{f}C)R_{i}} = -\frac{R_{f}}{R_{i}}\frac{1}{1+j\omega R_{f}C}
\end{split}
\end{equation}
At low frequencies, $|\frac{v_{out}}{v_{in}}(\omega)| \approx
\frac{R_{f}}{R_{i}}$, while at high frequencies
$|\frac{v_{out}}{v_{in}}(\omega)| \approx \frac{1}{\omega
R_{i}C}$.

The corner frequency will be where $\frac{R_{f}}{R_{i}}=
\frac{1}{\omega R_{i}C}$, or $\omega = 1/R_{f}C$.

Note that Bode plots are typically drawn showing $\pm 90$ degrees.
However, we consider phase to span 360 degrees, which is more correct
and also indicates the 180 degree phase shift.

Generally, the order of the filter, indicated in simple circuits by
the number of reactive components \textit{active in a particular
frequency band}, relates directly to the slope of the Bode plot in the
cutoff region or regions.

\end{document}
