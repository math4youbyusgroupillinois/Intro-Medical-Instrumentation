\documentclass[11pt]{report}
\usepackage{epsf}
\usepackage{amsmath}
\usepackage{amssymb}
\usepackage{palatino}
\usepackage[dvips]{graphics}
\usepackage{fancyhdr}
\parindent 0in
\parskip 1ex
\oddsidemargin  0in
\evensidemargin 0in
\textheight 8.5in
\textwidth 6.5in
\topmargin -0.25in

\pagestyle{fancy}
\fancyhead[LO]{\bf BME354L - Palmeri - Spring 2013}
\fancyhead[RO]{{\bf Second-Order Systems}}
\fancyfoot[C]{This work is licensed under a Creative Commons Attribution 3.0 Unported License.}


\begin{document}

% The next 7 lines were used when printing this section as a handout:
\setcounter{page}{1}
\setcounter{equation}{0}
\setcounter{figure}{0}
\begin{center}
\Large{\textbf{The Complex Exponential}}
\normalsize

\end{center}


We engineers like to describe signals as sinusoids, i.e.\
in terms of sine and cosine functions.  Why is this? 
For a number of reasons, including that:
\begin{itemize}
\item these functions efficiently describe the natural behavior
of a variety of natural and electrical systems of interest,
\item we utilize Fourier methods which allow us to express
any physically realizable signal as a weighted combination of sinusoids, and
\item we utilize Fourier methods to describe LSI (linear,
shift-invariant) systems, and by extension, to describe how such a
system responds to any physically realizable signal. The object is to
predict the output given some input.
\end{itemize}
A ``generalized'' sinusoid of amplitude $A$ and phase $\theta$ has the form:
\begin{equation}
f(t) = A \cos(\omega t + \theta)
\label{eq:trig}
\end{equation}
This seems like a perfectly good expression that is familiar to us
all.\\

\textbf{Why mess with it?} 

Because trigonometric forms such as that in Equation \ref{eq:trig}
become quite inconvenient when you start performing mathematical
transformations on them! Transformations of interest include advancing
or retarding the phase, integrating, differentiating, multiplying by
another signal, and of course, the Fourier Transform. Trigonometric
identities will take you only so far.

To avoid these difficulties, we adopt notation using the complex exponential,
$e^{j\omega t}$. By adopting a more complicated description of the
signal, we can actually make the mathematical transformations of
interest \textit{much} easier.

We recall Euler's Law, and use it to construct a ``generalized''
complex exponential:
\begin{equation}
\begin{split}
e^{j\phi} &= \cos\phi + j\sin\phi\\
Ae^{j(\omega t + \theta)} &=A[\cos(\omega t + \theta) + j\sin(\omega t + \theta)]\\
\end{split}
\end{equation}
The complication we have accepted in adopting this notation is that we
have converted a real signal into a complex signal. The imaginary part
has been introduced as a mathematical device with which to efficiently 
\textit{keep track of phase}. 

In describing the response of some circuit to a real sinusoidal input
signal, we begin by describing the input signal in complex notation,
perform operations on this signal to reflect the circuit's response, and
then solve for the system output by taking the real part of the
result.

The generalized complex exponential itself can be interpreted to
represent a complex vector rotating about the origin having length $A$
and instantaneous angle $\theta$ with respect to the real, positive
axis, as shown in Fig.~\ref{fig:phasor}.
\begin{figure}[h]
\centering
\epsfxsize=2.0in
\epsfbox{phasor.eps}
\caption{A phasor is a complex number that describes a particular
sinusoidal function represented with the complex exponential
$e^{j\omega t}$.  The phasor has magnitude $A$ and phase $\theta$. The
real component of this phasor is $Re[Ae^{j\theta}] = A \cos\theta$,
while the imaginary component is $Im[Ae^{j\theta}] = A \sin\theta$.
The generalized complex exponential function itself is of the form
$Ae^{j\theta}e^{j\omega t}$. At time $t=0$, the phasor diagram shows
the instantaneous value of the complex exponential function.}
\label{fig:phasor}
\end{figure}

The following identities follow:
\begin{equation}
\begin{split}
Re[Ae^{j\theta}] &= A \cos\theta\\
Im[Ae^{j\theta}] &= A \sin\theta\\
\theta &= \arctan\frac{Im[Ae^{j\theta}]}{Re[Ae^{j\theta}]}\\
|A| &= \sqrt{Re[Ae^{j\theta}]^2+Im[Ae^{j\theta}]^2}\\
e^{j0} = 1,~~~&~~~e^{j\pi/2} = j\\
e^{-j\pi/2} = -j,~~~&~~~e^{j\pi} = e^{-j\pi} =-1\\
\end{split}
\end{equation}
Mathematical manipulations include:
\begin{equation}
\begin{split}
\text{Phase shift of } \phi \text{ radians:~~~~~}& e^{j\phi}\cdot
Ae^{j(\omega t + \theta)} = Ae^{j(\omega t + \theta+\phi)}\\
\text{Differentiation:~~~~~}&
\frac{d}{dt}[Ae^{j(\omega t + \theta)}] = j\omega Ae^{j[\omega t + \theta]}\\
\end{split}
\end{equation}
\textbf{Example:} If a voltage function of the form $v_{C}(t) =
Re[Ae^{j\omega t}]$ is applied across a capacitor of value $C$, what is
the current through the capacitor?
\begin{equation}
\begin{split}
i_{C}(t) &= Re\big{[}C\frac{d}{dt}[Ae^{j\omega t}]\big{]}\\
\therefore ~~~~i_{C}(t) &= Re\big{[}j\omega C Ae^{j\omega t} = e^{j \pi/2}\omega C Ae^{j\omega t}\big{]}
~~~~(\text{current leads voltage by }90^\circ) \\
&= Re\bigg{[}Ae^{j\omega t} / \frac{1}{j\omega C}\bigg{]}\\
\therefore ~~~~i_{C}(t)&= Re\bigg{[}\frac{Ae^{j\omega t}}{Z_C}\bigg{]}~~~~\text{(Ohm's Law!)} 
\end{split}
\end{equation}

You \textbf{must} be comfortable intra-converting the sin/cosine,
complex exponential, polar, and complex number forms of a signal at some time
$t$.

The complex exponential times a phasor has the forms
$Ae^{j\theta}e^{j\omega t}$, $Ae^{j(\omega t + \theta)}$, and
$A[\cos(\omega t+\theta) + j\sin(\omega t+\theta)]$.

The complex phasor can also be represented in \textbf{polar form} $A\angle
\theta$ or \textbf{rectangular form} $A(\cos\theta + j \sin\theta)$.

\textbf{Recall} that for complex numbers $A$ and $B$:
\begin{equation}
\begin{split}
|AB| = |A||B|~~&~~|A/B| = |A|/|B|\\
\angle AB = \angle A + \angle B~~&~~\angle A/B = \angle A - \angle B\\
\end{split}
\end{equation}

These identities are very useful in simplifying the magnitude and
phase of a complicated transfer function.

\textbf{Note} that a $\pm 180^\circ$ or $\pm \pi$ phase shift
corresponds to a sign inversion.

\textbf{Note} that when using a calculator, the $\arctan$ function
only gives correct phases in the first and fourth quadrants, i.e.\ for 
a positive real part.  If the real part is negative, you must
shift the 1st or 4th quadrant answer by $180^\circ$ to get the correct
answer.

\textbf{Note} that Ohm's law and other circuit analysis principles can be applied to circuits with capacitors and
inductors using the complex exponential form of signals and the
complex impedance description of these components.

\textbf{Common mistakes} include forgetting to convert between radians and
degrees (related by the factor $180/\pi$), and between frequency
measured in radians/s and Hertz (related by the factor $1/(2 \pi)$).

\clearpage

\end{document}
